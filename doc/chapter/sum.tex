\section{设计总结}

\subsection{注意事项与编程技巧}

\subsubsection{使用 Procise 中遇到的问题}

我在使用 Procise 时, 遇到了一个非常奇怪的问题. 在例化 BRAM 时, 
Procise 默认按照面积最小的方式为我选择 BRAM 的资源组合方式, 然而其默认使用的
8K x 2 的 BRAM 在上板时遇到了问题, 无法正常工作. 经过我的排查, 我确定
问题在 BRAM 没有返回正常的数据. 我将 BRAM 的资源组合方式调整成 16K x 1 后, 
我的游戏奇迹般地能够正常显示了. 我怀疑此处 Procise 可能存在一些问题. 

\subsubsection{SpinalHDL 编程技巧}


当涉及到SpinalHDL的编程技巧时, 参数化和封装是两个重要的话题. 下面我将简要介绍一些与参数化和封装相关的技巧. 

首先, 参数化是指在设计中使用参数来灵活地定义和配置模块的行为. 通过参数化, 我们可以使设计更具通用性和可配置性, 从而提高代码的重用性和可维护性. 在SpinalHDL中, 我们可以定义参数, 并在模块实例化时传入具体的值. 这样, 同一个模块可以通过改变参数的值来实现不同的功能. 

其次, 封装是将模块或功能块包装成更高层次的接口, 隐藏内部实现细节, 并提供清晰的接口供其他模块使用. 通过封装, 我们可以简化设计的复杂性, 并提高代码的可读性和可维护性. 在SpinalHDL中, 我们可以使用Bundle或Vec等数据结构来封装信号和接口, 并通过定义合适的方法和函数来操作和访问这些封装好的接口. 

另外, SpinalHDL还提供了一些其他的技巧来优化和简化设计. 例如, 使用when语句来实现条件性的逻辑, 使用is语句来定义状态机的状态, 以及使用Enum来定义状态机的状态集合等. 这些技巧可以帮助我们更好地组织和描述设计的逻辑. 

必须要强调的是命名, 优秀的设计需要面对自己和他人, 需要兼顾可读性和性能. 我通过规范的命名来提高代码的可读性, 并添加合理的注释. 我相信即使再过几年, 我也能看懂我的代码. 

\subsection{心得体会}

\subsubsection{Procise 使用体会}

总体上而言, Procise 与 Vivado 的使用体验非常相似, 同时 Procise 能提供
比 Vivado 更快的综合, 布局布线和比特流生成速度. 同时 Procise 较 Vivado
更加轻量且不易卡顿. 但其相较于 Vivado 不足之处有以下两点: 

\begin{enumerate}
    \item 结果相对较差. 通过提升频率挖掘 Procise 和 Vivado 针对同一电路的频率极限, 
    发现 Procise 的频率极限相较于 Vivado 较低. 这也可能是由于 IO 管脚位置不同导致线网延迟差异所引起的.
    \item Procise 没有内置的前仿, 后仿工具, 使用较不方便. 
\end{enumerate}

总体而言, Procise 的功能比我预想的要强大, Procise 是一套优秀的国产 EDA 工具. 

\subsection{总结}

在这次项目中, 我使用了复旦微电子开发的FPGA开发板和Procise EDA工具, 结合敏捷开发工具SpinalHDL, 成功地开发了一个有趣的打地鼠游戏. 整个过程中, 我积极探索和尝试各种技术和工具, 同时也遇到了一些挑战和困难. 通过克服这些困难, 我不仅提高了自己的技术能力, 还深刻认识到了软硬件开发的复杂性和挑战. 

首先, 我要强调一下使用国产工具Procise的体验. Procise与Vivado相似, 几乎没有学习门槛, 并且其用户界面设计非常优秀. 我能够轻松上手并熟练运用Procise进行硬件开发, 这让我对国产EDA工具的发展和成熟感到非常欣慰. 

在项目的开发过程中, 我采用了敏捷开发的方法论, 这对我的工作效率和成果产生了显著的影响. 

为了加快开发速度, 我利用Verilator + Qt技术实现了硬件设计信号的读取和模拟显示器输出. 这种技术组合的运用让我能够在开发过程中进行快速迭代和调试, 在开发板组内共用, 资源受限的情况下极大地提高了我的设计速度. 

在面对困难和挑战时, 我始终保持积极的心态和勇于尝试的精神. 在项目的实施过程中, 我遇到了一些技术难题和逻辑复杂的设计要求. 然而, 我通过深入研究, 逐一解决了这些问题, 并不断提升自己的解决问题的能力. 

最后, 我要强调撰写设计报告对我的成长所起到的重要作用. 通过将设计过程和结果整理成报告的形式, 我不仅加深了对整个项目的理解, 还锻炼了自己的技术文档写作能力. 这项经历让我学会将复杂的技术概念和设计思路清晰地表达出来, 为将来的工作和学习提供了宝贵的经验. 

总的来说, 通过利用复旦微电子开发的FPGA开发板、Procise EDA工具和SpinalHDL敏捷开发工具, 我成功地完成了一个打地鼠游戏的设计项目. 在这个过程中, 我深入体验了国产工具的优势, 认识到敏捷开发方法对于项目效率的提升, 克服了各种挑战, 并通过撰写报告提升了自己的技术文档写作能力. 这次项目经验对于我个人和专业发展都具有重要的意义, 并为我今后的工作奠定了坚实的基础. 